\documentclass[a4paper,10pt]{article}
\usepackage[T1]{fontenc}
\usepackage[utf8]{inputenc}
\usepackage{geometry}
\usepackage{tabularx, longtable}
\usepackage{amsmath}
\usepackage[ngerman]{babel}
\usepackage[usenames,dvipsnames]{xcolor}
\usepackage{enumerate}
\usepackage{enumitem}
\setlist{nolistsep}
\usepackage{multirow}
%\usepackage{itemize}
% \usepackage{sans}
\usepackage{array}
\newcolumntype{C}[1]{>{\centering\arraybackslash$}p{#1}<{$}}
\usepackage{rotating}
\usepackage{pdflscape}
\usepackage{hyperref}

% \usepackage{mathpazo}% Palatino mit mathematischen Zeichen
\hyphenation{Prä-mi-en-frei-stel-lung}

%opening
\title{\textbf{Formelsammlung Tarif Lebensversicherung}}
\author{Reinhold Kainhofer (reinhold@kainhofer.com)}

\geometry{bottom=2cm,top=2cm,left=1.5cm,right=1.5cm}

\newcommand{\PreserveBackslash}[1]{\let\temp=\\#1\let\\=\temp}
\newcolumntype{v}[1]{>{\PreserveBackslash\centering\hspace{0pt}}p{#1}}

\newcommand{\markiert}[1]{\colorbox{Goldenrod}{\textcolor{red}{#1}}}
\newcommand{\orZero}[1]{\mathit{\left[#1\right]}}
\newcommand{\xn}{{\act[x]{n}}}


\setlength{\parindent}{0em}

\makeatletter
\DeclareRobustCommand{\act}[2][]{%
\def\arraystretch{0}%
\setlength\arraycolsep{0.5pt}%
\setlength\arrayrulewidth{0.5pt}%
\setbox0=\hbox{$\scriptstyle#1\ifx#1\empty{asdf}\else:\fi#2$}%
\begin{array}[b]{*2{@{}>{\scriptstyle}c}|}
\cline{2-2}%
\rule[1.25pt]{0pt}{\ht0}%
#1\ifx#1\empty\else:\fi & #2%
\end{array}%
}
\makeatother

\renewcommand{\labelitemi}{-)}
% \renewcommand[\itemsep}{0cm}
% \renewcommand{\labelitemsep}{0cm}
\itemsep0em


\begin{document}

\maketitle

% \begin{abstract}
% 
% \end{abstract}

Bemerkung: Sämtliche Barwerte werden rekursiv bestimmt, daher werden alle Formeln in ihrer rekursiven Form angegeben. Teilweise wird aus informativen Gründen davor die entsprechende Summenformel angegeben, diese wird jedoch nicht zur tatsächlichen Berechnung benutzt. 

\tableofcontents


\begin{landscape}
\section{Definitionen sämtlicher Variablen}

\subsection{Vertragsdetails (vertragsspezifische Werte)}

\newenvironment{deftab}{\begin{longtable}{p{1.7cm}p{13cm}p{9cm}}}{\end{longtable}}

\begin{deftab}
 $VS$ & Versicherungssumme & \texttt{contract\$params\$sumInsured}\\
 $\widetilde{VS}$ & Versicherungssumme nach außerplanmäßiger Prä"-mi"-en"-frei"-stel"-lung\\[0.5em]
 $x$ & Eintrittsalter der versicherten Person & \texttt{contract\$params\$age}\\
 $n$ & Versicherungsdauer & \texttt{contract\$params\$policyPeriod}\\
 $l$ & Aufschubdauer des Versicherungsschutzes & \texttt{contract\$params\$deferral}\\
 $m$ & Prämienzahlungsdauer & \texttt{contract\$params\$premiumPeriod}\\
 $k$ & Prämienzahlungsweise ($k$-tel jährlich) & \texttt{contract\$params\$premiumFrequency}\\
 $g$ & Garantiedauer (für Renten)  & \texttt{contract\$params\$guaranteed}\\
 $f$ & Prämienfreistellungszeitpunkt\\[0.5em]
 
 $YOB$ & Geburtsjahr der versicherten Person (bei Benutzung von Generationentafeln) & \texttt{contract\$params\$YOB}\\
 $Beg$ & Versicherungsbeginn (Datum, TODO)\\[0.5em]

 $q_{x+t}$ & einjährige Sterbewahrscheinlichkeit der versicherten Person (aus $YOB$ und $x$ bestimmt) & \texttt{contract\$params\$transitionProbabilities\$q}\\
 $i_{x+t}$ & einjährige Erkrankungswahrscheinlichkeit (z.B. Dread Disease, Krebs, etc.) & \texttt{contract\$params\$transitionProbabilities\$i} \\
 $p_{x+t}$ & einjährige Überlebenswahrscheinlichkeit als Aktiver, $p_{x+t} = 1-q_{x+t}-i_{x+1}$ & \texttt{contract\$params\$transitionProbabilities\$p} \\
 $\omega$  & Höchstalter gemäß der benutzten Sterbetafel  & \texttt{getOmega(tarif\$mortalityTable)} \\[0.5em]
 

 $k_{Ausz}$ & unterjährige Auszahlung der Erlebensleistungen (nur Renten) & \texttt{contract\$params\$benefitFrequency}\\
 $y$ & Eintrittsalter der 2. versicherten Person (TODO)\\
 
 
\end{deftab}

\subsection{Tarifdetails (identisch für alle Verträge)}
\begin{deftab}
 $i$ & Rechnungszins & \texttt{tarif\$i}\\[0.5em]
 $v$ & Diskontierungsfaktor $v=\frac1{1+i}$ & \texttt{tarif\$v}\\[0.5em]
 
 $\rho$ & Sicherheitszuschlag auf die Prämie & \texttt{tarif\$loadings\$security} \\
 $\rho^{RG}$ & Risikosumme (relativ zu DK) im Ablebensfall bei Prämienrückgewähr & \texttt{tarif\$premiumRefundLoading} \\
 $uz(k)$ & Unterjährigkeitszuschlag bei $k$-tel jährlicher Prämienzahlung (in \% der Prämie) & \texttt{tarif\$premiumFrequencyLoading}\\
 $O(k)$ & Ordnung der Unterjährigkeitsrechnung der Erlebenszahlungen (0./""1./""1,5./""2. Ordnung) & \texttt{tarif\$benefitFrequencyOrder}\\[0.5em]
 
\end{deftab}

\subsection{Leistungen}

\begin{deftab}
 $\Pi_t$ & (Netto-)Prämie zum Zeitpunkt $t$ (vorschüssig), $\Pi^{nachsch.}_t$ für nachschüssige Prämienzahlungsweise, normiert auf 1 & \texttt{contract\$cashFlows\$premiums\_advance}\\
 $\Pi^{nachsch.}_t$ & (Netto-)Prämie zum Zeitpunkt $t$ (nachschüssig), normiert auf 1 & \texttt{contract\$cashFlows\$premiums\_arrears}\\
 $PS_t$ &Bruttoprämiensumme bis zum Zeitpunkt $t$: $PS_t=\sum_{j=0}^t \Pi^a_t$\\
 $PS$ & Bruttoprämiensumme über die gesamte Vertragslaufzeit  & \texttt{contract\$premiumSum}\\[0.5em]
 
 $\ddot{e}_t$ & vorschüssige Erlebenszahlung zum Zeitpunkt $t$ (normiert auf 1) & \texttt{contract\$cashFlows\$survival\_advance}\\
 $e_t$ & nachschüssige Erlebenszahlung zum Zeitpunkt $t$ (normiert auf 1) & \texttt{contract\$cashFlows\$survival\_arrears}\\[0.5em]
 
 $\ddot{e}_t^*$ & vorschüssige garantierte Zahlung zum Zeitpunkt $t$ (normiert auf 1) & \texttt{contract\$cashFlows\$guaranteed\_advance}\\
 $e_t^*$ & nachschüssige garantierte Zahlung zum Zeitpunkt $t$ (normiert auf 1) & \texttt{contract\$cashFlows\$guaranteed\_arrears}\\[0.5em]
 
 $a_t$ & Ablebensleistung proportional zur Versicherungssumme (nachschüssig) & \texttt{contract\$cashFlows\$death\_sumInsured}\\
 $a_t^{(RG)}$ & Ablebensleistung für Prämienrückgewähr (normiert auf Prämie 1, relativ zu Prämienzumme $PS$) & \texttt{contract\$cashFlows\$death\_GrossPremium}\\[0.5em]
 $\widetilde{a}_t$ & Ablebensleistung nach außerplanmäßiger Prämienfreistellung proportional zur Versicherungssumme (nachschüssig) & \texttt{contract\$cashFlows\$death\_PremiumFree}\\
 
 $\overrightarrow{CF}^B_t$ & Leistungscashflow (relativ zur jeweiligen Basis, sowie vor"~/""nachschüssig) als Matrix dargestellt
\end{deftab}

\subsection{Kosten}
Mögliche Basen für die Kostenfestsetzung sind:
\begin{align*}
 \text{Basis} &= \begin{cases}\text{VS ... Versicherungssumme (\texttt{"{}SumInsured"})}\\\text{PS ... gesamte Prämiensumme  (\texttt{"{}SumPremiums"})}\\\text{BP ... Bruttojahresprämie  (\texttt{"{}GrossPremium"})}\end{cases} &
 \text{Dauer} &= \begin{cases}\text{1 ... einmalig bei Abschluss (\texttt{"{}once"})}\\\text{PD ... Prämienzahlungsdauer (\texttt{"{}PremiumPeriod"})}\\\text{Prf ... Nach Ablauf der Prämienzahlungsdauer (\texttt{"{}PremiumFree"})}\\\text{LZ ... gesamte Laufzeit (\texttt{"{}PolicyPeriod"})}\end{cases}
\end{align*}


\begin{deftab}
 $\alpha_t^{\text{Basis},\text{Dauer}}$ & Abschlusskostensatz relativ zu Basis über die angegebene Dauer & \texttt{tarif\$costs["{}alpha",,]}\\
 $Z_t^{\text{Basis},\text{Dauer}}$ & Zillmerkostensatz relativ zu Basis über die angegebene Dauer & \texttt{tarif\$costs["{}Zillmer",,]}\\
 $\beta_t^{\text{Basis},\text{Dauer}}$ & Inkassokostensatz relativ zu Basis über die angegebene Dauer & \texttt{tarif\$costs["{}beta",,]}\\
 $\gamma_t^{\text{Basis},\text{Dauer}}$ & Verwaltungskostensatz relativ zu Basis über die angegebene Dauer & \texttt{tarif\$costs["{}gamma",,]}\\
 $\widetilde{\gamma}_t^{\text{Basis},\text{Dauer}}$ & Verwaltungskostensatz nach außerplanmäßiger Prämienfreistellung relativ zu Basis über die angegebene Dauer & \texttt{tarif\$costs["{}gamma\_nopremiums",,]}\\[0.5em]
 
 $\alpha_t^{\text{Basis}}$ & Abschlusskosten-Cash Flow relativ zu Basis zu $t$ & \texttt{contract\$cashFlowsCosts[,"{}alpha", Basis]}\\
 $Z_t^{\text{Basis}}$ & Zillmerkosten-Cash Flow relativ zu Basis zu $t$  & \texttt{contract\$cashFlowsCosts[,"{}Zillmer", Basis]}\\
 $\beta_t^{\text{Basis}}$ & Inkassokosten-Cash Flow relativ zu Basis zu $t$  & \texttt{contract\$cashFlowsCosts[,"{}beta", Basis]}\\
 $\gamma_t^{\text{Basis}}$ & Verwaltungskosten-Cash Flow relativ zu Basis zu $t$  & \texttt{contract\$cashFlowsCosts[,"{}gamma", Basis]}\\
 $\widetilde{\gamma}_t^{\text{Basis}}$ & Verwaltungskosten-Cash Flow nach außerplanmäßiger Prämienfreistellung relativ zu Basis zu $t$  & \texttt{contract\$cashFlowsCosts[,"{}gamma\_nopremiums", Basis]}\\[0.5em]

 $\overrightarrow{CF}^C_t$ & Kostencashflows (relativ zur jeweiligen Basis) als Matrix dargestellt & \texttt{contract\$cashFlowsCosts["{}t",,]}
\end{deftab}

\subsection{Barwerte}

\begin{deftab}
 $P_\xn(t)$ & BW der zuk. Prämienzahlungen (mit Prämie 1) zum Zeitpunkt $t$ & \texttt{contract\$presentValues\$premiums}\\
 $E^*_\xn(t)$ & BW der zuk.  garantierten Zahlungen (mit VS 1) zum Zeitpunkt $t$ & \texttt{contract\$presentValues\$guaranteed}\\
 $E_\xn(t)$ & BW der zuk.  Erlebenszahlungen (mit VS 1) zum Zeitpunkt $t$ & \texttt{contract\$presentValues\$survival}\\
 $E^{(k)}_\xn(t)$ & BW der zuk.  Erlebenszahlungen (mit VS 1) bei $k$-tel jährlicher Auszahlung zum Zeitpunkt $t$ & \texttt{contract\$presentValues\$survival}\\
 $\alpha(k)$, $\beta(k)$ & Unterjährigkeitskorrektur bei $k$-tel jährlicher Auszahlung\\
 $A_\xn(t)$ & BW der zuk.  Ablebensleistungen (mit VS 1) zum Zeitpunkt $t$ & \texttt{contract\$presentValues\$death\_SumInsured}\\
 $A_\xn^{prf}(t)$  & BW der zuk.  Ablebensleistungen (mit VS 1) zum Zeitpunkt $t$ nach Prämienfreistellung & \texttt{contract\$presentValues\$death\_GrossPremium}\\
 $A^{(RG)}_\xn(t)$ & BW der zuk.  Ablebensleistungen aus Prämienrückgewähr (mit BP 1) zum Zeitpunkt $t$ & \texttt{contract\$presentValues\$death\_PremiumFree}\\[0.5em]
 
 $PV^{B1}_\xn(t)$ & BW aller zuk.  Leistungen (ohne Prämienrückgewähr) zum Zeitpunkt $t$ & \texttt{contract\$presentValues\$benefits}\\[1em]
 $PV^B_\xn(t)$ & BW aller zuk.  Leistungen (inkl. Prämienrückgewähr) zum Zeitpunkt $t$ & \texttt{contract\$presentValues\$benefitsAndRefund}\\[1em]
 
\end{deftab}
 KOSTEN (TODO)

\subsection{Prämien und Prämienzerlegung}

\begin{deftab}
 $\Pi^1_\xn$ & Nettoprämie auf VS 1 & \texttt{contract\$premiums[["{}unit.net"]]}\\
 $\Pi_\xn^{1,Z}$ & Zillmerprämie auf VS 1 & \texttt{contract\$premiums[["{}unit.Zillmer"]]}\\
 $\Pi_\xn^{1,a}$ & Bruttoprämie (``adequate'' bzw. ``expense-load premium'') auf VS 1 & \texttt{contract\$premiums[["{}unit.gross"]]}\\[0.5em]
 $\Pi_\xn$ & Nettoprämie & \texttt{contract\$premiums[["{}net"]]}\\
 $\Pi_\xn^Z$ & Zillmerprämie & \texttt{contract\$premiums[["{}Zillmer"]]}\\
 $\Pi_\xn^a$ & Bruttoprämie (``adequate'' bzw. ``expense-load premium'') & \texttt{contract\$premiums[["{}gross"]]}\\[0.5em]
 
 $\Pi_\xn^\alpha$ & $\alpha$-Kostenprämie (Abschlusskostenprämie)\\
 $\Pi_\xn^\beta$ & $\beta$-Kostenprämie (Inkassokostenprämie)\\
 $\Pi_\xn^\gamma$ & $\gamma$-Kostenprämie (Verwaltungskostenprämie)\\[0.5em]

 
 $\Pi_\xn^\text{inv.}$ & Inventarprämie (Netto- plus Verwaltungskostenprämie)\\[1em]
 
 $\Pi_\xn^s$ & Sparprämie (zum Aufbau des Nettodeckungskapitals investierter Teil der Prämie)\\
 $\Pi_\xn^r$ & Risikoprämie (zur Deckung des einjährigen Ablebensrisikos benutzter Teil der Prämie)\\[0.5em]
 
 $\Pi_\xn^v$ & verrechnete Prämie (Bruttoprämie inkl. Rabatte, Zuschläge, Stück"-kosten und Steuer) & \texttt{contract\$premiums[["{}written"]]}\\
 $\Pi_\xn^\text{tax}$ & Versicherungssteuer & \texttt{contract\$premiums[["{}tax"]]}\\
\end{deftab}

\subsection{Absolutwerte der Cashflows und Barwerte}

TODO

\subsection{Rückstellungen}

\begin{deftab}
 $_{t}V_\xn$ & Nettodeckungskapital zum Zeitpunkt $t$  & \texttt{contract\$reserves[,"net"]}\\
 $_{t}V^a_\xn$ & Brutto-Deckungskapital (``adequate'' bzw. ``expense-loaded reserve'') & \texttt{contract\$reserves[,"adequate"]}\\
 $_{t}V^Z_\xn$ & Zillmerreserve bei Zillmerung & \texttt{contract\$reserves[,"Zillmer"]}\\
 $_{t}V^\alpha_xn$ & Abschlusskostenreserve & \texttt{contract\$reserves[,"TODO"]}\\
 $_{t}V^\beta_xn$ & Inkassokostenreserve (typischerweise $=0$)& \texttt{contract\$reserves[,"TODO"]}\\
 $_{t}V^\gamma_xn$ & Verwaltungskostenreserve & \texttt{contract\$reserves[,"gamma"]}\\
 $_{t}V^{Umr}_\xn$ & Umrechnungsreserve (Basis für Vertragskonvertierungen und Prämienfreistellung), inkl. anteiliger Abschlusskostenrückerstattung bei Beendigung innerhalb von fünf Jahren& \texttt{contract\$reserves[,"reduction"]}\\
 $AbskErh(t)$ & Anteilsmäßige Rückerstattung der Abschlusskosten bei Kündigung innerhalb von fünf Jahren& \texttt{contract\$reserves[,"alphaRefund"]}\\[1em]
 
 $BilRes_{t+u}$  & Bilanzreserve (Interpolation aus  $_{t}V^{x}_\xn$ und $_{t+1}V^{x}_\xn$)
\end{deftab}

\subsection{Werte nach außerplanmäßiger Prämienfreistellung}

Werte nach außerplanmäßiger Prämienfreistellung werden durch ein $\widetilde{\quad}$ über dem jeweiligen Symbol angezeigt.
 




\end{landscape}
\pagebreak

\section{Kosten}

\subsection{Abschlusskosten (\texorpdfstring{$\alpha$}{alpha}-Kosten) / Zillmerkosten (Z-Kosten)}
\begin{itemize}
 \item Einmalig (bei Vertragsabschluss)
 \begin{itemize}
   \item \markiert{an Versicherungssumme}
   \item \markiert{an Brutto-Prämiensumme}\footnote{Entspricht Einmalprämie bei Einmalerlag}
   \item an Barwert der Versicherungsleistungen (z.B. Rentenbarwert) % HP PV5F & PV6F
 \end{itemize}
 \item Laufend (während Prämienzahlungsdauer)\footnote{Bei Einmalerlag sind einmalige $\alpha$-Kosten und laufende $\alpha$-Kosten auf die Prämie während der Prämienzahlungsdauer ident.}
 \begin{itemize}
   \item \markiert{an Bruttoprämie}
   \item an Brutto-Prämiensumme
 \end{itemize}
 \item Laufend (über gesamte Laufzeit des Vertrags)
 \begin{itemize}
   \item an Bruttoprämie
   \item an Brutto-Prämiensumme
 \end{itemize}
 
\end{itemize}

% TODO: Check the following tariffs:
%   - GW PV5F & PV6F

\subsection{Inkassokosten (\texorpdfstring{$\beta$}{beta}-Kosten)}

\begin{itemize}
 \item Laufend an Bruttoprämie während Prämienzahlungsdauer (einmalig bei Einmalerlag)
\end{itemize}


\subsection{Verwaltungskosten (\texorpdfstring{$\gamma$}{gamma}-Kosten)}

Laufend während der gesamten Laufzeit verrechnet:

\begin{itemize}
 \item \markiert{an Versicherungssumme (prämienpflichtig)}
 \item \markiert{an Versicherungssumme (planmäßig/außerplanmäßig prämienfrei)}
 \item an Leistungsbarwert / Rentenbarwert (=Deckungskapital) (prämienfrei) % GW PS0-9 & PV0-9
 \item an Prämiensumme (prämienpflichtig) (=am Rentenbarwert zu Vertragsbeginn bei sof.beg.LR mit EE)
 \item an Prämiensumme (planmäßig/außerplanmäßig prämienfrei)
 
 \item am Ablösekapital während Aufschubzeit % ER III B-PZV Single
 \item \markiert{an jeder Erlebenszahlung/Rente (während Liquiditätsphase)} % ER III B-PZV Single
 \item am Deckungskapital % SV I5U
\end{itemize}

% TODO: Check the following tariffs:
%   - GW PS0-9 & PV0-9
%   - ER III B-PZV Single

\subsection{Stückkosten \textit{StkK}}

\begin{itemize}
 \item Stückkosten (Absolutbetrag) $StkK$ pro Jahr während Prämienzahlungsdauer (bzw. einmalig bei Einmalprämie)
\end{itemize}


\subsection{Übersicht}

Die häufigsten Kostentypen sind \markiert{markiert}.

% % \begin{longtable}{||l|v{3cm}|v{3cm}|v{3cm}|v{3cm}||}\hline\hline
% %  & an VS & an PS & an JBP\footnote{während der gesamten Prämienzahlungsdauer} & \\ \hline
% %  
% % Abschluss $\alpha$ & \markiert{$\alpha^{VS}_{j}$ (einmalig)} & $\alpha^{PS}_{j}$ (LZ)\footnote{evt. mit jährlicher faktorieller Aufwertung (evt. mit Obergrenze)} & $\alpha^{BP}_{j}$ (LZ)\linebreak \markiert{$\alpha_{3a,j}$ (PrD)} & \\ \hline
% % Zillmer $z$ & \markiert{$z^{VS}_{j}$} & \markiert{$z^{PS}_{j}$} & & \\ \hline
% % Inkasso $\beta$ & & & \markiert{$\beta_{j}$ (PrD)} & \\ \hline
% % Verwaltung $\gamma$ & \markiert{$\gamma^{VS}_{j}$ (PrD)}\linebreak \markiert{$\gamma^{VS,fr}_{j}$ (nach PrD)}\linebreak \markiert{$\gamma^{frei}_{j}$ (PrFrei)} & $\gamma^{PS}_{j}$ (LZ/PrD) & & $\gamma^{BP,PrD}_{j}$ (an ErlZ, LZ) \\\hline\hline
% % % \\[-0.9em]\hline
% % % 
% % % Stückkosten $K$ & 
% % % \\\hline\hline
% % % 
% % %  
% % \end{longtable}
% % 

\begin{longtable}{||lr|v{2cm}|v{2cm}|v{2cm}|v{3cm}||}\hline\hline
Typ & Dauer & an VS & an PS & an JBP\footnote{während der gesamten Prämienzahlungsdauer} & \\ \hline
 

Abschluss $\alpha$ & einmalig      & \markiert{$\alpha^{VS,once}$} &                                                                                                    &                            & \\ 
                   & Prämiendauer  &                              &                                                                                                    & \markiert{$\alpha^{BP,PrD}$} & \\ 
                   & Prämienfrei   &                              &                                                                                                    &                            & \\ 
                   & Vertragsdauer &                              & $\alpha^{PS,LZ}$\footnote{evt. mit jährlicher faktorieller Aufwertung, evt. mit Obergrenze)} & $\alpha^{BP,LZ}$          & \\ \hline
Zillmer $z$ & einmalig      & \markiert{$z^{VS,once}$} & \markiert{$z^{PS,once}$} & & \\ 
            & Prämiendauer  &                         &                         & & \\ 
            & Prämienfrei   &                         &                         & & \\ 
            & Vertragsdauer &                         & $z^{PS,LZ}$ & & \\ \hline
Inkasso $\beta$ & einmalig      & & &                              & \\ 
                & Prämiendauer  & & & \markiert{$\beta^{BP,PrD}$} & \\ 
                & Prämienfrei   & & &                              & \\ 
                & Vertragsdauer & & &                              & \\ \hline
Verwaltung $\gamma$ & einmalig      &  & & & \\
                    & Prämiendauer  & \markiert{$\gamma^{VS,PrD}$}    & $\gamma^{PS,PrD}$ & & \\
                    & Prämienfrei   & \markiert{$\gamma^{VS,fr}$} &                   & & $\gamma^{BP,Erl}$ (an ErlZ) \\
                    & Vertragsdauer & $\gamma^{VS,LZ}$           & $\gamma^{PS,LZ}$ & &  \\\hline

\multirow{4}{2.5cm}{{Verwaltung $\tilde{\gamma}$ (außerplanm. prämienfrei)}} & einmalig      & & & & \\
                    & Prämiendauer  & & & & \\
                    & Prämienfrei   & & & & \\
                    & Vertragsdauer & \markiert{$\tilde{\gamma}^{VS,LZ}$} & & & \\
\hline\hline
\end{longtable}

\subsection{Kosten-Cashflows}
Jede Kostenart ($\alpha$/Zillmer/$\beta$/$\gamma$/$\tilde{\gamma}$) und Bemessungsgrundlage (VS/{}PS/{}JBP) erzeugt aus den verschiedenen Kostendauern einen Cash-Flow-Vektor in folgender Art, der diskontiert den gesamten Kostenbarwert der jeweiligen Kostenart und Bmgl. liefert:
\begin{equation*}
 X_t^{Bmgl} = 
 \begin{cases}
 X^{Bmgl,once} + X^{Bmgl,PrD} + X^{Bmgl,LZ} & \text{für $t=0$} \\
 \hphantom{X^{Bmgl,once} + \;\,} X^{Bmgl,PrD} + X^{Bmgl,LZ} & \text{für $0<t\leq m$} \\
 \hphantom{X^{Bmgl,once} + \;\,} X^{Bmgl,fr} \;\;\; + X^{Bmgl,LZ} & \text{für $m<t\leq n$}
 \end{cases}
\end{equation*}



% \subsubsection{Stückkosten:}
% \begin{longtable}{lp{13cm}}
% $K_1$\dots & Stückkosten einmalig bei Vertragsabschluss\\
% $K$\dots & Stückkosten jährlich während Prämienzahlungsdauer\\
% $K_{LZ}$\dots & Stückkosten jährlich während der gesamten Laufzeit\\
% \end{longtable}

\pagebreak

\section{Normierte Cashflows (auf Einheitswerte)}

% \begin{sidewaystable}
\begin{center}
 

% \rotatebox{90}{
\begin{tabular}{lp{6.5cm}v{1.6cm}v{1.5cm}v{1.5cm}v{1cm}v{1cm}}
& Beschreibung & LR & ALV & ELV & T-F & DD\\\hline\\[0.01em]
$pr_t$  \dots          & Prämienzahlungen (vorschüssig) zu $t$ & $t<m$ & $t<m$ & $t<m$ & $t<m$ & $t<m$ \\
$PS$                   & Prämiensumme, $PS=\sum_{t=0}^n pr_t$\\[0.5em]\hline\\[0em]

$\ddot{e}_t$  \dots    & Erlebenszahlungen vorschüssig zu $t$ & $(l+g)...n$ & 0 & $t=n$ & 0 & 0 \\
$e_t$  \dots           & Erlebenszahlungen nachschüssig zu $t+1$& $(l+g)...n$ & 0 & $t=n$ & 0 & 0 \\
$\ddot{e}_t^{*}$ \dots & garantierte Zahlungen vorschüssig zu $t$ & $(l+g)...n$ & 0 & 0  & $t=n$ & 0 \\
$e_t^{*}$  \dots       & garant. Zahlungen nachschüssig zu $t+1$ & $(l+g)...n$ & 0 & 0  & 0 & 0 \\[0.5em]\hline\\[0em]

$a_t$  \dots           & Ablebenszahlung zu $t+1$ & 0 & $l...n$ & 0 & 0 & 0 \\
$d_t$  \dots           & Zahlung zu $t+1$ bei Eintritt der Erkrankung zwischen $t$ und $t+1$ & 0 & 0 & 0 & 0& $l...n$ \\[0.5em]\hline\\[0em]
$a_t^{(RG)}$           & Ablebenszahlungen für PRG zu $t+1$ (Ableben im Jahr $t$) & $\min(t+1, m, f)$ & 0 & $\min(t+1,m,f)$ & 0 & 0 \\[0.5em]\hline

\end{tabular}
% }
\end{center}
% \end{sidewaystable}

Die Cash-Flows können auch in Matrixform dargestellt werden:
\begin{align*}
%
 \overrightarrow{CF}^L_t &= \left(
 \begin{matrix} % TODO: Add an indication that the first row is in advance, the second in arrears
pr_t & \ddot{e}_t^{*} & \ddot{e}_t & 0 & 0 & 0 \\
pr^{(nachsch)}_t & e_t^{*} & e_t & a_t & d_t & a_t^{(RG)}\\
\end{matrix}
 \right)
% 
&
 \overrightarrow{CF}^K_t &= \left(
 \begin{matrix}
\alpha^{(VS)}_t & \alpha^{(PS)}_t  & \alpha^{(BP)}_t \\
z^{(VS)}_t & z^{(PS)}_t  & -\\
- & - & \beta_t \\
\gamma^{(VS)}_t & \gamma^{(PS)}_t & -\\
\widetilde{\gamma}^{(VS)}_t & - & -\\  
 \end{matrix}
 \right)
\end{align*}


\pagebreak

\section{Barwerte}

\subsection{Prämienbarwert}

\begin{align*} 
P_\xn(t) &= \sum_{j=t}^n pr_{t+j} \cdot v^{j-t}  \cdot {}_{j-t}p_{x+t}\\
	  &= pr_{t} + v \cdot p_{x+t} \cdot P_\xn(t+1)
\end{align*}

\subsection{Barwert garantierter Zahlungen:}
Garantierte Erlebensleistungen (wenn Aufschubzeit überlebt wurde):
% TODO!
\begin{align*} 
 E^{*}_\xn(t) &= \begin{cases}
		    {}_{l-t}p_{x+t} \cdot v^{l-t} \cdot \sum_{j=l}^{n} \left\{\ddot{e}^{*}_{j-t}+ v\cdot e^*_{j-t}\right\} v^{j-t} & \text{für $t<l$ (Aufschubzeit)}\\
		    \sum_{j=t}^{n} \left\{\ddot{e}^{*}_{j-t}+ v\cdot e^*_{j-t}\right\} v^{j-t} & \text{für $t\ge l$ (Liquiditätsphase)}
                 \end{cases}\\
   &= \ddot{e}_t^{*} + \left\{E^{*}_\xn(t+1) + e_t^*\right\}\cdot v \cdot \begin{cases}
	    1 & \text{für $t<l$ (Aufschubzeit)}\\
            p_{x+t} & \text{für $t\ge l$ (Liquiditätsphase)}
       \end{cases}
\end{align*}


\subsection{Erlebensleistungsbarwert:}
% 1. Person:
\begin{align*} 
E_\xn(t) &= \sum_{j=t}^n \left(\ddot{e}_{t+j} \cdot v^{j-t}  {}_{j-t}p_{x+t} + e_{t+j} \cdot v^{j+1-t} {}_{j+1-t}p_{x+t} \right)\\
	  &= \ddot{e}_{t} + v \cdot p_{x+t} \cdot \left\{ e_t + E_\xn(t+1)\right\}\\
% \intertext{2. Person:}
% E2_{\act[y]{n}}(t) &= \ddot{e}_{t} + v \cdot p_{y+t} \cdot \left\{ e_t + E2_{\act[y]{n}}(t+1)\right\}\\
% \intertext{gemeinsam:}
% E12_{\act[x,y]{n}}(t) &= \ddot{e}_{t} + v \cdot p_{x+t} \cdot p_{y+t} \cdot \left\{ e_t + E12_{\act[x,y]{n}}(n,t+1)\right\}\\
\end{align*}

% \intertext{Garantierte Erlebensleistungen (wenn Aufschubzeit überlebt wurde):}
% % \begin{align*} 
%  E^{*}_\xn(t) &= \begin{cases}
% 		    {}_{l-t}p_{x+t} \cdot v^{l-t} \cdot \sum_{j=l}^{n} \left\{\ddot{e}^{*}_{j-t}+ v\cdot e^*_{j-t}\right\} v^{j-t} & \text{für $t<l$ (Aufschubzeit)}\\
% 		    \sum_{j=t}^{n} \left\{\ddot{e}^{*}_{j-t}+ v\cdot e^*_{j-t}\right\} v^{j-t} & \text{für $t\ge l$ (Liquiditätsphase)}
%                  \end{cases}\\
%    &= \ddot{e}_t^{*} + \left\{E^{*}_\xn(t+1) + e_t^*\right\}\cdot v \cdot \begin{cases}
% 	    1 & \text{für $t<l$ (Aufschubzeit)}\\
%             p_{x+t} & \text{für $t\ge l$ (Liquiditätsphase)}
%        \end{cases}
% \end{align*}
\subsection{Unterjährige Auszahlung der Erlebenszahlungen}

Analog zu (bei konstanter Rente)
\begin{align*}
 \ddot{a}^{(m)}_\xn &= \ddot{a}_x^{(m)} - {}_np_x \cdot v^n \cdot \ddot{a}_{x+n}^{(m)}\\
 \ddot{a}_x^{(m)} &= \alpha(m) \ddot{a}_x - \beta(m)
\end{align*}
mit 
\begin{align*}
 \alpha(m)&= \frac{d \cdot i}{d^{(m)} \cdot i^{(m)}} & \beta(m) &= \frac{i-i^{(m)}}{d^{(m)} \cdot i^{(m)}}
\end{align*}
und $d = \frac{i}{1+i}$, $i^{(m)} = m \cdot \left((1+i)^{1/m} -1\right)$ und $d^{(m)} = i^{(m)} / \left(1+i^{(m)}/m\right)$ bzw. approximativ mit
\begin{center}
\begin{tabular}{rccccc}
& 0.Ord. & 1.Ord. & 1,5-te Ord. & 2.Ord. & \\\hline
$\alpha(m)$ = & 1 &  & & $+\frac{m^2-1}{12 m^2} i^2$ & ...\\
$\beta(m)$ = & $\frac{m-1}{2m}$ & $+ \frac{m^2-1}{6 m^2}i$ & $\left[+ \frac{1-m^2}{12\cdot m^2}\cdot i^2\right]$ &$+ \frac{1-m^2}{24 m^2}i^2$ & ...\\[0.3em]\hline 
\end{tabular}
\end{center}

ergibt sich auch für allgemeine unterjährige Erlebenszahlungen $\ddot{e}_t$ eine Rekursionsgleichung.

\subsubsection{Vorschüssige \textit{m}-tel jährliche Auszahlung der Erlebensleistungen}


\begin{align*}
E^{(m)}_\xn(t) &= \ddot{e}_t \cdot \ddot{a}_{\act[x+t]{1}}^{(m)} + v \cdot p_{x+1} \cdot E^{(m)}_\xn(t+1)\\
  &= \ddot{e}_t \cdot \left\{\alpha(m)  - \beta(m) \cdot \left(1-p_{x+t} \cdot v\right)\right\} + v\cdot p_{x+t} \cdot E^{(m)}_\xn(t+1)
\end{align*}


\subsubsection{Nachschüssige \textit{m}-tel jährliche Auszahlung der Erlebensleistungen}

\begin{align*}
E^{(m)}_\xn(t) &= e_t \cdot a_{\act[x+t]{1}}^{(m)} + v \cdot p_{x+t} \cdot E^{(m)}_\xn(t+1)\\
   &= e_t\cdot\left\{\alpha(m) - \left(\beta(m)+\frac1m\right)\cdot\left(1-p_{x+t} v\right)\right\} + v\cdot p_{x+t} \cdot E^{(m)}_\xn(t+1)
\end{align*}

\subsubsection{Allgemeine \textit{m}-tel jährliche Auszahlung der Erlebensleistungen}


\begin{align*}
E^{(m)}_\xn(t) = &\ddot{e}_t \cdot \left\{\alpha(m)  - \beta(m) \cdot \left(1-p_{x+t} \cdot v\right)\right\} +  \\
 &e_t\cdot\left\{\alpha(m) - \left(\beta(m)+\frac1m\right)\cdot\left(1-p_{x+t} v\right)\right\} + \\
 &v\cdot p_{x+t} \cdot E^{(m)}_\xn(t+1)
\end{align*}




\subsection{Ablebensbarwert}


\begin{align*} 
\intertext{\subsubsection*{Prämienpflichtiger Ablebensbarwert}}
A_\xn(t) &= \sum_{j=t}^n {}_{j-t}p_{x+t} \cdot q_{x+j} \cdot v^{j-t+1} \cdot a_{j} \\
         &= q_{x+t} \cdot v \cdot a_t + p_{x+t} \cdot v \cdot A_\xn(t+1)\\
%
\intertext{\subsubsection*{Prämienfreier Ablebensbarwert}}
A^{(prf)}_\xn(t) &= q_{x+t} \cdot v \cdot a^{(prf.)}_t + p_{x+t} \cdot v \cdot A^{(prf.)}_\xn(t+1)\\
%
\intertext{\subsubsection*{Barwert der gesamten Prämienrückgewähr (an BP)}}
A^{(RG)}_\xn(t) &= q_{x+t} \cdot v \cdot a^{(RG)}_t + p_{x+t} \cdot v \cdot A^{(RG)}_\xn(t+1)\\
%
\intertext{\subsubsection*{Barwert der bisher bereits erworbenen Prämienrückgewähr (an BP)}}
A^{(RG-)}_\xn(t) &= a^{(RG)}_{t-1} \cdot A^{(RG-,1)}_\xn(t)\\
A^{(RG-,1)}_\xn(t) &= q_{x+t} \cdot v \cdot 1 + p_{x+t} \cdot v \cdot A^{(RG-,1)}_\xn(t+1)\\
%
\intertext{\subsubsection{Barwert der künftig noch zu erwerbenden Prämienrückgewähr (an BP)}}
A^{(RG+)}_\xn(t) &= A^{(RG)}_\xn(t) - A^{(RG-)}_\xn(t)
\intertext{Dieser Wert ist rekursiv nicht darstellbar.}
\end{align*}

\subsection{Disease-Barwert}
\begin{align*} 
A^d_\xn(t) &= i_{x+t} \cdot v \cdot d_t + p_{x+t} \cdot v \cdot A^d_\xn(t+1)\\
\end{align*}

\subsection{Leistungsbarwert}

\begin{align*}
BW^L_\xn(t) &= E_\xn(t) + A_\xn(t) + A^d_\xn(t) &\text{ohne Prämienrückgewähr}\\
BW^{L,RG}_\xn(t) &= BW^L_\xn(t) + (1+\rho^{RG}) \cdot A^{(RG)}_\xn(t)\cdot BP_\xn &\text{inkl. Prämienrückgewähr}
\end{align*}



\subsection{Kostenbarwerte}

\begin{align*}
\intertext{Abschlusskostenbarwerte:}
AK^{(VS)}_\xn(t) &= \alpha^{VS}_{t} + v \cdot p_{x+t} \cdot AK^{(VS)}_\xn(t+1) \\
AK^{(PS)}_\xn(t) &= \alpha^{PS}_{t} + v \cdot p_{x+t} \cdot AK^{(PS)}_\xn(t+1) \\
AK^{(BP)}_\xn(t) &= \alpha^{BP}_{t} + + v \cdot p_{x+t} \cdot AK^{(BP)}_\xn(t+1) \\
\intertext{Zillmerkostenbarwerte:}
ZK^{(VS)}_\xn(t) &= z^{VS}_{t} + v \cdot p_{x+t} \cdot ZK^{(VS)}_\xn(t+1) \\
ZK^{(PS)}_\xn(t) &= z^{PS}_{t} + v \cdot p_{x+t} \cdot ZK^{(PS)}_\xn(t+1) \\
\intertext{Inkassokostenbarwerte:}
IK^{(BP)}_\xn(t) &= \beta^{BP}_{t} + v \cdot p_{x+t} \cdot IK^{(BP)}_\xn(t+1) \\
\intertext{Verwaltungskostenbarwerte:}
VK^{(VS)}_\xn(t) &= \gamma^{VS}_{t} + v \cdot p_{x+t} \cdot VK^{(VS)}_\xn(t+1) \\
VK^{(PS)}_\xn(t) &= \gamma^{PS}_{t} + v \cdot p_{x+t} \cdot VK^{(PS)}_\xn(t+1) \\
\widetilde{VK}^{(VS)}_\xn(t) &= \widetilde{\gamma}^{VS}_{t} + v \cdot p_{x+t} \cdot \widetilde{VK}^{(VS)}_\xn(t+1) \\
\end{align*}


\subsection{Darstellung der Barwerte in Vektor-/Matrixform}
Die Leistungs- und Kostenbarwerte können (wie auch die Cashflows zu einem Zeitpunkt) 
in Matrixform dargestellt werden (aus Gründen der Übersichtlichkeit wird hier bei 
allen Termen der Subscript $\xn$ unterlassen):
\begin{align*}
  \overrightarrow{BW}^L(t) &= \left(
 \begin{matrix}
    P(t), & %E^{Gar}(t), & E(t), & A(t), 
    BW^L(t), & A^{(RG)}(t)
 \end{matrix}
 \right)
% 
 &
%  
 \overrightarrow{BW}^K(t) &= \left(
 \begin{matrix}
AK^{(VS)}(t) & AK^{(PS)}(t)  & AK^{(BP)}(t) \\
% 
ZK^{(VS)}(t) & ZK^{(PS)}(t)  & -\\
% 
- & - & IK^{(BP)}(t) \\
% 
VK^{(VS)}(t) & VK^{(PS)}(t) & -\\
\widetilde{VK}^{(VS)}(t) & - & -\\  
 \end{matrix}
 \right)
% 
\end{align*}



\pagebreak

\section{Prämien}

\subsection{Nettoprämie:}
\begin{align*}
NP_\xn &= \frac{%E_\xn(0) + A_\xn(0) + \left(1+\rho^{RG}\right) \cdot A^{(RG)}_\xn(0) \cdot BP_\xn}
BW^{L}_\xn(0) + \left(1+\rho^{RG}\right) \cdot A^{(RG)}_\xn(0) \cdot BP_\xn}
{P_\xn(0)} \cdot \left(1+\rho\right)
\end{align*}


\subsection{Zillmerprämie (gezillmerte Nettoprämie):}
% TODO: Sind die beta-Kosten proportional zu BP oder zur ZP??? I.e. ist \beta im Nenner mit BP oder im Zähler?
\begin{align*}
ZP_\xn &= 
\frac{%
%   NP_\xn\cdot P_\xn(0) +
  \overbrace{\left(BW^{L}_\xn(0) + \left(1+\rho^{RG}\right) \cdot A^{(RG)}_\xn(0) \cdot BP_\xn\right) \cdot (1+\rho)}^{=NP_\xn\cdot P_\xn(0)} +
  ZK^{(VS)}_\xn(0) + 
  ZK^{(PS)}_\xn(0) \cdot BP_\xn \cdot PS + 
  ZK^{(BP)}_\xn(0)\cdot BP_\xn
%
}{P_\xn(0)}
\end{align*}

Varianten:
\begin{itemize}
 \item % REFERENCE: GE Tarife!
     $\beta$- und $\gamma$-Kosten auch in die Zillmerprämie eingerechnet. Einziger Unterschied zur Bruttoprämie ist dann, dass nur die Zillmerkosten statt der $\alpha$-Kosten aufgeteilt werden.
\begin{align*}
ZP_\xn &= 
\left[%
%   NP_\xn\cdot P_\xn(0) + 
  BW^{L,RG}_\xn(0) \cdot (1+\rho) +
  \left(ZK^{(VS)}_\xn(0) + IK^{(VS)}_\xn(0) + VK^{(VS)}_\xn(0)\right) +  \right.\\
  &\qquad\left.\left(ZK^{(PS)}_\xn(0) + IK^{(PS)}_\xn(0) + VK^{(PS)}_\xn(0)\right) \cdot BP_\xn \cdot PS + \right.\\
  &\qquad\left.\left(ZK^{(BP)}_\xn(0) + IK^{(BP)}_\xn(0) + VK^{(BP)}_\xn(0)\right) \cdot BP_\xn
%
\right] / %
\left(P_\xn(0)\right)
\end{align*}
 
 \item % REFERENCE: ME
    Prämienrückgewähr proportional zu Zillmerprämie (für Berechnung der Zillmerprämie):
\begin{align*}
ZP_\xn &= \frac{BW^L_\xn(0) + \left(1+\rho^{RG}\right) \cdot A^{(RG)}_\xn(0) \cdot ZP_\xn}{P_\xn(0)} \cdot \left(1+\rho\right)\\
ZP_\xn &= \frac{BW^L_\xn(0)}{P_\xn(0) - \left(1+\rho^{RG}\right)\cdot A^{(RG)}_\xn(0)\cdot \left(1+\rho\right)} \cdot \left(1+\rho\right)
\end{align*}
 
\end{itemize}


\subsection{Bruttoprämie:}
\begin{align*}
BP_\xn &= \frac%
{ 
   BW^L_\xn(0)\cdot\left(1+\rho\right) + 
   \left( AK^{(VS)}_\xn(0) + IK^{(VS)}_\xn(0) + VK^{(VS)}_\xn(0) \right)
}%
{
   P_\xn(0) - 
   A^{(RG)}_\xn \left(1+\rho^{RG}\right) \left(1+\rho\right) - 
         AK^{(BP)}_\xn - IK^{(BP)}_\xn - VK^{(BP)}_\xn - 
   \left(AK^{(PS)}_\xn + IK^{(PS)}_\xn + VK^{(PS)}_\xn\right) PS
}
\end{align*}
Wie man deutlich sehen kann, ist die Kostenursache ($\alpha$, $\beta$ oder $\gamma$) für die Prämienbestimmung irrelevant. Es werden die Barwerte aller drei Kostenarten jeweils bei der entsprechenden Bemessungsgrundlage aufaddiert.
% 
\subsection{Ablebensleistung im Jahr \textit{t}:}
\begin{align*}
Abl(t) &= \left\{a_t + a^{(RG)}_t \cdot BP_\xn\right\} \cdot VS
\end{align*}

\subsection{Koeffizienten in Vektorschreibweise}
Für die Berechnung der Prämien können die Koeffizienten der jeweiligen Barwerte auch mittels der Vektor-/Ma\-trix\-schreib\-wei\-se dargestellt werden (siehe Tabelle \ref{PVCoeff}).
\begin{sidewaystable}
\centering

% \newenvironment{benarray}{\big(\begin{array}{*4{C{3.4em}}C{12em}}}{\end{array}\big)}
\newenvironment{benarray}{\big(\begin{array}{C{4em}C{5em}C{12em}}}{\end{array}\big)}
\newenvironment{costarray}{\left(\begin{array}{*3{C{4.7em}}}}{\end{array}\right)}
% TODO: laufende alpha-Kosten
\begin{tabular}{||ll|c|c||}\hline\hline
 & & Leistungen & Kosten \\ \hline\hline
 Terme & &  
%   $\begin{benarray}P_\xn(t) & E^{Gar}_\xn(t) & E_\xn(t) & A_\xn(t)  & A_\xn^{(RG)}(t)\end{benarray}$
  $\begin{benarray}P_\xn(t) & BW^L_\xn(t)  & A_\xn^{(RG)}(t)\end{benarray}$
 &
 $\begin{costarray}
AK^{(VS)}_\xn(t) & AK^{(PS)}_\xn(t)  & AK^{(BP)}_\xn(t) \\
ZK^{(VS)}_\xn(t) & ZK^{(PS)}_\xn(t)  & -\\
- & - & IK_\xn(t) \\
VK^{(VS)}_\xn(t) & VK^{(PS)}_\xn(t) & -\\
VK^{frei}_\xn(t) & - & -\\  
 \end{costarray}$
\\\hline\hline


Nettoprämie & Zähler & 
  $\begin{benarray}0 & 1+\rho & \left(1+\rho^{RG}\right)\cdot BP_\xn \cdot \left(1+\rho\right)\end{benarray}$
 & -
\\

 & Nenner & 
  $\begin{benarray} 1 & 0 & 0 \end{benarray}$
 & - 
\\\hline


Zillmerprämie & Zähler & 
  $\begin{benarray}0 & 1+\rho & \left(1+\rho^{RG}\right)\cdot BP_\xn \cdot \left(1+\rho\right)\end{benarray}$
 &
 $\begin{costarray}
0 & 0 & 0 \\
1 & BP_\xn\cdot PS & BP_\xn\\
\orZero{1} & \orZero{BP_\xn\cdot PS} & \orZero{BP_\xn} \\
\orZero{1} & \orZero{BP_\xn\cdot PS} & \orZero{BP_\xn} \\
0 & 0 & 0 \\
 \end{costarray}$
\\

 & Nenner & 
  $\begin{benarray} 1 & 0 & 0\end{benarray}$
 &
-
\\\hline

Bruttoprämie & Zähler & 
   $\begin{benarray}0 & 1+\rho & 0\end{benarray}$
 &
 $\begin{costarray}
1 & 0 & 0 \\
0 & 0 & 0\\
1 & 0 & 0 \\
1 & 0 & 0\\
0 & 0 & 0\\
 \end{costarray}$
\\

 & Nenner & 
  $\begin{benarray}1 & 0 & -(1+\rho)\cdot(1+\rho^{RG}) \end{benarray}$
 &
 $\begin{costarray}
0 & -PS & -1 \\
0 & 0   & 0\\
0 & -PS & -1 \\
0 & -PS & -1\\
0 & 0   & 0\\
 \end{costarray}$
\\

\hline\hline

\end{tabular}
\label{PVCoeff}
\caption{Koeffizienten der einzelnen Barwerte zur Berechnung der Prämien}
\end{sidewaystable}



\section{Zuschläge und Abschläge, Vorgeschriebene Prämie}

\begin{longtable}{p{4cm}p{11cm}}
 $oUZu$ \dots & Zuschlag für Vertrag ohne ärztliche Untersuchung\\
 $SuRa=SuRa(VS)$ \dots & Summenrabatt (von Höhe der VS abhängig)\\
 $VwGew$ \dots & Vorweggewinnbeteiligung in Form eines \%-uellen Rabattes auf die Bruttoprämie\\
 $StkK$ \dots & Stückkosten pro Jahr (während Prämienzahlungsdauer, einmalig bei Einmalprämien)\\
 $PrRa=PrRa(BP)$ \dots & Prämienrabatt (von Höhe der Bruttoprämie abhängig)\\
 $VwGew_{StkK}$ \dots & Vorweggewinnbeteiligung in Form eines Rabattes auf die Prämie nach Zu-/Abschlägen (insbesondere nach Stückkosten)\\
 $PartnerRa$ \dots & Partnerrabatt auf Prämie nach Zu-/Abschlägen (z.B. bei Abschluss mehrerer Verträge), additiv zu $VwGew_{StkK}$\\
 
 $uz(k)$ \dots & Zuschlag für unterjährige Prämienzahlung ($k$ mal pro Jahr)
 \begin{equation*}
  uz(k)=\left.\begin{cases}uk_1 & \text {für jährliche}\\uk_2 & \text {für halbjährliche} \\ uk_4 & \text{für quartalsweise}\\uk_{12} & \text{für monatliche}\end{cases}\right\} \text{Prämienzahlung}
 \end{equation*}\\
 
 
 $VSt$ \dots & Versicherungssteuer (in Österreich 4\% oder 11\%) \\

\end{longtable}


Vorgeschriebene Prämie:
\begin{multline*}
PV_\xn = \left\{ (BP_\xn + oUZu - SuRa) \cdot VS \cdot (1-VwGew) + StkK\right\} \cdot \\ \left(1-PrRa-VwGew_{StkK}-PartnerRa\right)\cdot \frac{1+uz(k)}{k} \cdot (1+VSt)
% 
\end{multline*}

\section{Absolute Cash-Flows und Barwerte}

TODO


\pagebreak

\section{Rückstellungen und Reserven}

\subsection{Deckungskapital / Reserve}

\subsubsection{Nettodeckungskapital prämienpflichtig:}
\begin{align*}
V_\xn(t) &= \left\{BW^L_\xn(t)\cdot(1+\rho) - NP_\xn\cdot P_\xn(t)\right\} \cdot VS
\end{align*}

\subsubsection{Zillmerreserve prämienpflichtig:}
TODO!
\begin{align*}
V_\xn(t) &= \left\{BW^L_\xn(t)\cdot(1+\rho) - ZP_\xn\cdot P_\xn(t)\right\} \cdot VS =\\
 &= \left\{BW^L_\xn(t)\cdot(1+\rho) - NP_\xn\cdot P_\xn(t) - ZK_\xn(0) \cdot BP_\xn(t) \cdot \frac{P_\xn(t)}{P_\xn(0)}\right\} \cdot VS \\
\end{align*}

\subsubsection{Reserve prämienpflichtig:}
Entspricht bei Zillmerung der Zillmerreserve
\begin{align*}
V_\xn(t) &= \left\{BW^L_\xn(t)\cdot(1+\rho) - ZP_\xn\cdot P_\xn(t)\right\} \cdot VS \\
\end{align*}

\subsubsection{Bruttoreserve prämienpflichtig:}

\begin{align*}
V^{(b)}_\xn(t) &= \left\{BW^L_\xn(t)\cdot(1+\rho) +  - ZP_\xn\cdot P_\xn(t)\right\} \cdot VS \\
\end{align*}

\subsection{Verwaltungskostenreserve:}
\begin{align*}
V^{VwK}_\xn(t) &= \left\{ VK^{(VS)}_\xn(t) - \left(\frac{VK^{(VS)}_\xn(0)}{P_\xn(0)}\right) \cdot P_\xn(t)\right\} \cdot VS\\
\end{align*}

\subsection{Reserve prämienfrei:}
\begin{align*}
V^{frei}_\xn(t) &= \left\{(E_\xn(t) + A1_\xn(t))\cdot\widetilde{VW} + TODO \cdot \min(f,m) \cdot BP_\xn(x,n)\cdot VS\right\} \cdot (1+\rho) \\
\end{align*}

\subsection{Verwaltungskostenreserve prämienfrei:}
\begin{align*}
V^{WvK,frei}_\xn(t) &= VK4_\xn(t) \cdot \widetilde{VS}
\end{align*}

\section{Spar- und Risikoprämie}


\begin{equation*}
 P_\xn(t) = SP_\xn(t) + RP_\xn(t)
\end{equation*}


\subsection{Sparprämie}

\begin{align*}
SP_\xn(t) &= V_\xn(t+1) \cdot v - V_\xn(t) + \left(\ddot{e}_t + v\cdot e_t\right)\cdot VS
\end{align*}

\subsection{Risikoprämie}
\begin{align*}
RP_\xn(t) &= v\cdot q_{x+t} \cdot \left\{Abl(t) - V_\xn(t+1)\right\} 
\end{align*}


\section{Bilanzreserve}

\begin{tabular}{lp{14cm}}
$BegDatum$ \dots & Beginndatum des Vertrags\\
$BilDatum$ \dots & Bilanzstichtag  des Unternehmens\\
$baf$ \dots & Bilanzabgrenzungsfaktor (Jahresanteil zwischen Abschlussdatum und Bilanzstichtag)
\begin{itemize}
 \item 30/360: $baf=\frac{Monat(BilDatum+1)-Monat(BegDatum)+1}{12} \mod 1$
 \item Taggenau: $baf=\frac{BilDatum-BegDatum+1}{TageImJahr(BilDatum)} \mod 1$
 \item etc.
\end{itemize}
\\
\end{tabular}


\subsection{prämienpflichtig}
% Bilanzabgrenzungsfaktor 
Bilanzreserve für Versicherungsleistungen:
\begin{align*}
BilRes^{(L)}_\xn(t) &= (1-baf)\cdot V_\xn(t) + baf \cdot V_\xn(t+1)\\
\intertext{Verwaltungskosten-Bilanzreserve:}
BilRes^{(VwK)}_\xn(t) &= (1-baf)\cdot V^{(VwK)}_\xn(t) + baf \cdot V^{(VwK)}_\xn(t+1)\\
\intertext{Gesamte Bilanzreserve:}
BilRes_\xn(t) &= BilRes^{(L)}_\xn(t) + BilRes^{(VwK)}_\xn(t)\\
\intertext{\subsection{prämienfrei}
Bilanzreserve für Versicherungsleistungen, prämienfrei:}
BilRes^{(L),frei}_\xn(t) &= (1-baf)\cdot V^{frei}_\xn(t) + baf \cdot V^{frei}_\xn(t+1)\\
\intertext{Verwaltungskosten-Bilanzreserve, prämienfrei:}
BilRes^{(VwK),frei}_\xn(t) &= (1-baf)\cdot V^{VwK,frei}_\xn(t) + baf \cdot V^{VwK,frei}_\xn(t+1)\\
\intertext{Gesamte Bilanzreserve, prämienfrei:}
BilRes^{frei}_\xn(t) &= BilRes^{(L),frei}_\xn(t) + BilRes^{(VwK),frei}_\xn(t)\\\
\end{align*}


\pagebreak

\section{Prämienfreistellung und Rückkauf}

Verteilung der $\alpha$-Kosten auf $r$ Jahre für den Rückkauf bzw. die Vertragskonversion 
ist nicht bei allen Tarifen oder in allen Jurisdiktionen vorgesehen. => FLAG

\subsection{Umrechnungsreserve}
Sowohl Prämienfreistellung als auch Rückkauf starten von der Umrechnungsreserve, die sich aus der 
Zillmerreserve, den Kostenrückstellungen sowie der Verteilung der $\alpha$-Kosten auf 5 Jahre ergibt:
\begin{align*}
 V_\xn^{Umr} &= \left(V_\xn(t) + V^{VwK}_\xn(t) + AbsKErh(t)\right)\cdot (1-VwGew(TODO))
\end{align*}
wobei $AbsKErh(t)$ die anteilsmäßige Rückzahlung der Abschlusskosten bei Rückkauf innerhalb der ersten 
$n(=5)$ Jahre gemäß \S 176 öVersVG bezeichnet:
\begin{align*}
AbskErh(t) &= \max\left(\sum_{j=0}^t Zillm(j) - \frac{t}{5} \sum_{j=0}^n Zillm(j), 0\right)&& \text{(Abschlusskostenerhöhungsbetrag)}\\
Zillm(t) &= z^{(VS)}_t + z^{(BP)} \cdot BP_\xn + z^{(PS)}_t \cdot BP_\xn \cdot \sum_{j=0}^n pr_j &&\text{(Zillmerprämienanteil/-cashflow im Jahr $j$)}
\end{align*}

Varianten:
\begin{itemize}
 \item Verteilung auf 5 Jahre nicht linear ($t/5$), sondern als 5-jährige Leibrente bewertet, deren Rest noch ausständig ist.
 \begin{align*}
   AbskErh(t) &= \max\left(\sum_{j=0}^t Zillm(j) - \left(1-\frac{\ddot{a}_{\act[x+t]{r-t}}}{\ddot{a}_{\act[x]{r}}}\right) \frac{t}{5} \sum_{j=0}^n Zillm(j), 0\right)
 \end{align*}
 \item Bei zahlreichen Tarifen wird die Abschlusskostenerhöhung erst NACH dem Rückkaufsabschlag addidiert, 
    sodass diese Erhöhung nicht vom Abschlag betroffen ist => FLAG
\end{itemize}

\subsection{Rückkaufswert (prämienpflichtig)}

Zahlreiche Tarife sind NICHT rückkaufsfähig => FLAG

\begin{align*}
 Rkf(t) &= f(V_\xn^{Umr}, ...) 
\end{align*}

Die Abschläge von der Umrechnungsreserve auf den Rückkaufswert sind im Allgemeinen nicht 
standardisiert, sondern variieren je nach Versicherungsunternehmen stark. Mögliche
Abschläge sind:

\subsubsection*{Prozentualer Rückkaufsabschlag}
Prozentualer Abschlag auf die Umrechnungsreserve, z.B. $2\%$ oder $5\%$:
$RkfFakt=0.95$
\begin{align*}
f(V_\xn^{Umr}, ...) = RkfFakt \cdot V_\xn^{Umr} \text{\quad mit $RkfFakt=0.98$ oder $0.95$}
\end{align*}

\subsubsection*{Lineare Erhöhung des prozentualen Rückkaufsabschlags}
\begin{align*}
f(V_\xn^{Umr}, ...) &= RkfFakt(t) \cdot V_\xn^{Umr} \\
RkfFakt(t) &= min(k_1 + t \cdot \delta k; k_2) \text{\quad mit z.B. $k_1=0.9$, $\delta k = 0.005$ und $k_2=0.98$}
\end{align*}
Alternativ:
\begin{equation*}% ME L12PLK3Z
 RkfFakt(t) = \begin{cases}                
0.95 & 1\leq t \leq 3\\
0.95 + 0.003\cdot(t-3) & 3<t\leq 13\\
0.98 & 13<t
\end{cases}
\end{equation*}


\subsubsection*{Prozentualer Abschlag mit Mindestabschlag}
% GW GDV1
\begin{align*}
f(V_\xn^{Umr}, ...) &= min\left(0.95 \cdot V_\xn^{Umr}, Abl(t), V_\xn^{Umr}-0.15 \cdot BP_\xn \cdot VS\cdot (1-VwGew)\right)
\end{align*}

% GW ER11
\begin{align*}
f(V_\xn^{Umr}, ...) &= min\left(0.95 \cdot V_\xn^{Umr}, Abl(t)\right)
\end{align*}


\subsubsection*{Prozentualer Abschlag mit Mindestabschlag (Mindesttodesfallsumme als Grenze)}
% HP/GW:
\begin{align*}
f(V_\xn^{Umr}, ...) &= min(0.95 \cdot V_\xn^{Umr}, MTS(m,t)) \\
MTS(m,t) &= ...
\end{align*}


\subsubsection*{Abschlag proportional zum Deckungskapital}
% GE
\begin{align*}
f(V_\xn^{Umr}, ...) &= V_\xn^{Umr}\cdot \left(s_f + \max(0.97-s_f, 0) \cdot \frac{V_\xn^{Umr}}{VS}\right) \\
s_f &=\begin{cases}0.92 & \text{ für $t<\max(10, n-5)$}\\ 1 & \text{sonst}\end{cases}
\end{align*}


TODO: Weitere mögliche Rückkaufsabschläge rausfinden

\subsection{Stornogebühr bei Rückkauf}

Manche Tarife sehen eine fixe Stornogebühr bei Rückkauf (z.B. nur in den ersten 24 Monaten) vor:
% GW
\begin{equation*}
 StoGeb = \min\left(\max\left(0.15 \cdot PV(x,n) \cdot\frac{pz}{1-uz(pz)} \cdot\frac1{1+VSt}, 30 \right), 300\right)
\end{equation*}
Ansonsten: $StoGeb=0$.



\subsection{Prämienfreistellung}

Der Vertrag wird zum Zeitpunkt $f$ prämienfrei gestellt, d.h. ab $f$ wird keine Prämie mehr bezahlt, 
die Höhe des Versicherungsschutzes bestimmt sich aus dem zu $f$ vorhandenen Deckungskapital und den 
Kostenreserven (Umrechnungsreserve). Bei Prämienrückgewähr wird nur die tatsächlich bezahlte Prämiensumme rückgewährt.

Aus
\begin{align*}
 V_\xn^{Umr}(f) - StoGeb %= \\
 = \underbrace{BW^L_\xn(f)\cdot\left(1+\rho\right)\cdot \widetilde{VS} + BW^{RG,frei}_\xn(f)\cdot \left(1+\rho\right) \cdot BP_\xn \cdot VS}_{=V^{frei}_\xn(f)} + \underbrace{VK^{frei}_\xn(f)}_{=V_\xn^{VwK,frei}(f)}
%  =  V^{frei}_\xn(f) + V_\xn^{VwK,frei}(f)
\end{align*}
mit
\begin{align*}
BW^{RG,frei}_\xn(f) &= A^{(RG)}_\xn(t) \cdot \underbrace{\sum_{j=0}^{f-1} pr_j}_{\substack{=\min\left(f,m\right)\text{ bei }\\\text{lfd. konst. Prämie}}} && \text{(BW zukünftiger Prämienrückgewähr)}
\end{align*}
ergibt sich die neue Versicherungssumme $\widetilde{VS}(f)$ nach Prämienfreistellung zum Zeitpunkt $f$:
\begin{equation*}
\widetilde{VS}(f) = \frac
  { V_\xn^{Umr}(f) - BW^{RG,frei}_\xn(f)\cdot (1+\rho)\cdot BP_\xn \cdot VS - StoGeb}
  {BW^L_\xn(f)\cdot(1+\rho) + VK^{frei}_\xn(f)}
\end{equation*}

\subsection{Reserven nach außerplanmäßiger Prämienfreistellung}

\subsubsection*{Nettodeckungskapital außerplanmäßig Prämienfrei zu $f$}

\begin{align*}
V^{(n),prf,f}_\xn(t) &= \left\{BW^{L,prf}_\xn(t)\cdot(1+\rho)\right\} \cdot \widetilde{VS(f)}
\end{align*}

\subsubsection{Reserve außerplanmäßig prämienfrei:}

\begin{align*}
V_\xn^{prf,f}(t) &= \left\{BW^{L,pr}_\xn(t)\cdot(1+\rho) + BW^{RG,frei,f}_{\act[x]{x}}(t)\right\} \cdot \widetilde{VS(f)} \\
\end{align*}

\subsection{Verwaltungskostenreserve außerplanmäßig prämienfrei:}
\begin{align*}
V^{VwK,prf,f}_\xn(t) &= \left\{ VK^{(VS), prf.}_\xn(t) + VK^{(PS),prf.}_\xn(t)\cdot PS(f)\right\} \cdot \widetilde{VS(f)}\\
\end{align*}


TOCHECK:
\subsection{Reserve prämienfrei:}
\begin{align*}
V^{frei}_\xn(t) &= \left\{(E_\xn(t) + A1_\xn(t))\cdot\widetilde{VW} + TODO \cdot \min(f,m) \cdot BP_\xn(x,n)\cdot VS\right\} \cdot (1+\rho) \\
\end{align*}

\subsection{Verwaltungskostenreserve prämienfrei:}
\begin{align*}
V^{WvK,frei}_\xn(t) &= VK4_\xn(t) \cdot \widetilde{VS}
\end{align*}

\subsection{Umrechnungsreserve außerplanmäßig prämienfrei}
\begin{align*}
 V_\xn^{Umr, prf, f}(t) &= \left(V^{prf,f}_\xn(t) + V^{VwK,prf,f}_\xn(t)\right)\cdot (1-VwGew(TODO))
\end{align*}





\end{document}
